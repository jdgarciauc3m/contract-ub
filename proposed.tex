\section{Proposed solutions}
\label{sec:proposed}

\subsection{Avoiding the undefined behavior}

This paper proposes to avoid the undefined behavior by clarifying the semantics of every
build mode in regards of both evaluation of conditions and assumption of those
conditions. For assumption of conditions the clarification needs to address the
case where continuation is disabled and when the continuation is enabled.

To define such semantics, the following simple principles are proposed to be followed:

\begin{itemize}

\item A contract that, in a given build mode, is not evaluated cannot be used
for any kind of assumption. This leads to modes where only contracts that have
been checked are used for assumptions and avoiding in this way the identified
paths towards undefined behavior.

\item Moreover, contracts that have been checked can only be assumed if the
continuation mode has been disabled. Otherwise, such assumptions cannot be made.
Note, that no special provision is needed as the application of general rules
of conditional statements would derive this behavior as illustrated in previous
sections.

\item Axiom contract are considered as if they had been evaluated when they are
	enabled. Otherwise, they are ignored.  Note that in any case, they need
		to have a valid syntax, although expressions in an axiom are
		allowed to contain invocations to declared but not defined
		functions. If an axiom contains any invocation that is declared
		but not defined the axiom is ignored.

\end{itemize}

Below, the exact semantics of each build level are identified when 
they are applied
to each contract level.

\subsubsection{Build mode}

The build mode can be any of the following four: \textmark{off},
\textmark{default}, 
\textmark{audit}. This build level affeccts which checks are evaluated at run-time.

\vspace{1em}

\begin{tabular}{|l|c|c|c|}
\hline
\emph{contract-level} & \multicolumn{3}{c|}{Build mode}\\
\cline{2-4}
& \textmark{off} & \textmark{default} & \textmark{audit} \\
\hline
\hline
	\cppid{axiom} & not checked & not checked & not checked\\
\hline
	\cppid{audit} & not checked & not checked & checked\\
\hline
	\cppid{default} & not checked & checked & checked\\
\hline
\end{tabular}

\vspace{1em}

Note, that only \cppid{audit} and \cppid{default} checks are affected
by the build mode.

\subsection{Allowing assumption of axioms}

This paper proposes that an axiom mode is added. The axiom mode can be either
\textmark{off} or \textmark{on}.

\begin{itemize}
  \item When the axiom mode is \textmark{off} an implementation is not
        allowed to make any assumption.
  \item When the axiom mode is \textmark{on} an implementation is
        allowed to assume the axiom.
\end{itemize}

\subsection{Continuation mode}

This paper proposes to remove the continuation mode.

\begin{itemize}

  \item An invocation to a returning violation handler results
	in a call to \cppid{std::terminate} after executing the violation 
	handler.
\end{itemize}

\subsection{A new syntax for continuation}

This paper proposes to add syntax for a new \cppkey{continue} adjective that can
be applied to any \cppkey{default} or \cppkey{audit} contract.

\begin{lstlisting}
[[assert continue: predicate]]; // Default predicate with continuation
[[assert default continue: predicate]]; // Same as above

[[assert audit continue: predicate]]; // Aduit predicate with continuation
\end{lstlisting}

\subsection{Summary of semantics}

In this section a summary of the build options and their semantics is presented.

The translation is controlled by the following options:

\begin{itemize}
  \item Build mode: \cppid{off}, \cppid{default}, and \cppid{audit}.
  \item Axiom mode: \cppid{off}, \cppid{on}.
\end{itemize}

\begin{enumerate}

\item Build-mode=\cppid{off}, Axiom-mode=\cppid{off}.
\begin{itemize}
  \item No check is performed.
  \item No assumption is made.
\end{itemize}

\item Build-mode=\cppid{off}, Axiom-mode=\cppid{on}.
\begin{itemize}
  \item No check is performed.
  \item Checks with \cppkey{axiom} level are assumed.
\end{itemize}

\item Build-level=\cppid{default}, Axiom-mode=\cppid{off}.
\begin{itemize}
  \item Checks with \cppkey{default} level are evaluated and assumed.
  \item Checks with \cppkey{audit} are neither performed nor assumed.
  \item Code for \cppkey{default} contract checks 
	  assumes that the violation handler does not return.
  \item Code for \cppkey{default continue} contract checks
	  does not assume that the violation handler does not return.
  \item No \cppkey{axiom} is assumed.
\end{itemize}

\item Build-level=\cppid{default}, Axiom-mode=\cppid{on}.
\begin{itemize}
  \item Checks with \cppkey{default} level are evaluated and assumed.
  \item Checks with \cppkey{audit} are neither performed nor assumed.
  \item Code for \cppkey{default} contract checks 
	  assumes that the violation handler does not return.
  \item Code for \cppkey{default continue} contract checks
	  does not assume that the violation handler does not return.
  \item Checks with \cppkey{axiom} level are assumed.
\end{itemize}

\item Build-level=\cppid{audit}, Axiom-mode=\cppid{off}.
\begin{itemize}
  \item Checks with \cppkey{default} level are evaluated and assumed.
  \item Checks with \cppkey{audit} level are evaluated and assumed.
  \item Code for \cppkey{default} and \cppkey{audit} contract checks 
	  assumes that the violation handler does not return.
  \item Code for \cppkey{default continue} and \cppkey{audit continue} contract 
	  checks does not assume that the violation handler does not return.
  \item No \cppkey{axiom} is assumed.
\end{itemize}

\item Build-level=\cppid{audit}, Axiom-mode=\cppid{on}.
\begin{itemize}
  \item Checks with \cppkey{default} level are evaluated and assumed.
  \item Checks with \cppkey{audit} level are evaluated and assumed.
  \item Code for \cppkey{default} and \cppkey{audit} contract checks 
	  assumes that the violation handler does not return.
  \item Code for \cppkey{default continue} and \cppkey{audit continue} contract 
	  checks does not assume that the violation handler does not return.
  \item Checks with \cppkey{axiom} level are assumed.
\end{itemize}

\end{enumerate}
