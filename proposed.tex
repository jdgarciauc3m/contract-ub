\section{Proposed solutions}

\subsection{Avoiding the undefined behavior}

We propose to avoid the undefined behavior by clarifying the semantics of every
build mode in regards of both evaluation of conditions and assumption of those
conditions. For assumption of conditions the clarification needs to address the
case where continuation is disabled and when the continuation is enables.

To define such semantics we propose to follow the following simple principles:

\begin{itemize}

\item A contract that, in a given build mode, is not evaluated cannot be used
for any kind of assumption. This leads to modes where only contracts that have
been checked are used for assumptions and avoiding in this way the identified
paths towards undefined behavior.

\item Moreover, contracts that have been checked can only be assumed if the
continuation mode has been disabled. Otherwise, such assumptions cannot be made.
Note, that no special provision is needed as the application of general rules
of conditional statements would derive this behavior as illustrated in previous
sections.

\item Axiom contract are considered as if they had been evaluated 
when they are enabled. Otherwise, they are ignored.
Note that in any case, they need to have a valid syntax.

\end{itemize}

Below, we identify what are the exact semantics of each build level when applied
to each contract level.

\subsubsection{Build mode}

The build mode can be any of the following three: \textmark{off}, \textmark{default}, 
\textmark{audit}. This build level affeccts which checks are evaluated at run-time.
\vspace{1em}. Note, that only \cppid{audit} and \cppid{default} checks are affected
by the build mode.

\begin{tabular}{|l|c|c|c|}
\hline
\emph{contract-level} & \multicolumn{3}{c|}{Build mode}\\
\cline{2-4}
& \textmark{off} & \textmark{default} & \textmark{audit} \\
\hline
\hline
\cppid{audit} & no & no & yes\\
\hline
\cppid{default} & no & yes & yes\\
\hline
\end{tabular}

\subsubsection{Continuation mode}

The continuation mode can be either \textmark{off} or \textmark{on}.
Note that the generated code for contracts check might be different depending
on the continuation mode:

\begin{itemize}
  \item When the continuation mode is \cppid{off} an implementation is
        allowed to assume that an invocation to the violation handler
        will not return an generate code accordingly.
  \item When the continuation mode is \cppid{on} an implementation is 
        not allowed to perform such assumption, which will result in
        different code generation.
\end{itemize}

\subsection{Allowing assumption of axioms}

We propose that we add an axiom mode. The axiom mode can be either
\textmark{off} or \textmark{on}.

\begin{itemize}
  \item When the axiom mode is \textmark{off} an implementation is not
        allowed to make any assumption.
  \item When the axiom mode is \textmark{on} an implementation is
        allowed to assume the axiom.
\end{itemize}

Note that when axiom mode is \textmark{on}, an axiom condition establishes
an assumption barrier. That assumption barrier prevents that the condition
is not backwards propagated.

\subsection{Summary of semantics}

In this section we summarize the build options and their semantics.

The translation is controlled by the following options:

\begin{itemize}
  \item Build mode: \cppid{off}, \cppid{default}, and \cppid{audit}.
  \item Axiom mode: \cppid{off}, \cppid{on}.
  \item Continuation mode: \cppid{off}, \cppid{on}.
\end{itemize}

\begin{enumerate}

\item Build-mode=\cppid{off}, Axiom-mode=\cppid{off}, continuation=\cppid{off|on}.
\begin{itemize}
  \item No check is performed.
  \item No assumption is made.
\end{itemize}

\item Build-mode=\cppid{off}, Axiom-mode=\cppid{on}, continuation=\cppid{off|on}.
\begin{itemize}
  \item No check is performed.
  \item Checks with \cppid{axiom} level are assumed.
\end{itemize}

\item Build-level=\cppid{default}, Axiom-mode=\cppid{off}, continuation=\cppid{off}
\begin{itemize}
  \item Checks with \cppid{default} level are evaluated and assumed.
  \item Code for contract checks assumes that the violation handler does not return.
  \item No assumption is made.
\end{itemize}

\item Build-level=\cppid{default}, Axiom-mode=\cppid{on}, continuation=\cppid{off}
\begin{itemize}
  \item Checks with \cppid{default} level are evaluated and assumed.
  \item Code for contract checks assumes that the violation handler does not return.
  \item Checks with \cppid{axiom} level are assumed.
\end{itemize}

\item Build-level=\cppid{default}, Axiom-mode=\cppid{off}, continuation=\cppid{on}
\begin{itemize}
  \item Checks with \cppid{default} level are evaluated but not assumed.
  \item Code for contract checks assumes that the violation handler may return.
  \item No assumption is made.
\end{itemize}

\item Build-level=\cppid{default}, Axiom-mode=\cppid{on}, continuation=\cppid{on}
\begin{itemize}
  \item Checks with \cppid{default} level are evaluated but not assumed.
  \item Code for contract checks assumes that the violation handler may return.
  \item Checks with \cppid{axiom} level are assumed.
\end{itemize}

\item Build-level=\cppid{audit}, Axiom-mode=\cppid{off}, continuation=\cppid{off}
\begin{itemize}
  \item Checks with \cppid{audit} level are evaluated and assumed.
  \item Checks with \cppid{default} level are evaluated and assumed.
  \item Code for contract checks assumes that the violation handler does not return.
  \item No assumption is made.
\end{itemize}

\item Build-level=\cppid{audit}, Axiom-mode=\cppid{on}, continuation=\cppid{off}
\begin{itemize}
  \item Checks with \cppid{audit} level are evaluated and assumed.
  \item Checks with \cppid{default} level are evaluated and assumed.
  \item Code for contract checks assumes that the violation handler does not return.
  \item Checks with \cppid{axiom} level are assumed.
\end{itemize}

\item Build-level=\cppid{audit}, Axiom-mode=\cppid{off}, continuation=\cppid{on}
\begin{itemize}
  \item Checks with \cppid{audit} level are evaluated but not assumed.
  \item Checks with \cppid{default} level are evaluated but not assumed.
  \item Code for contract checks assumes that the violation handler may return.
  \item No assumption is made.
\end{itemize}

\item Build-level=\cppid{audit}, Axiom-mode=\cppid{on}, continuation=\cppid{on}
\begin{itemize}
  \item Checks with \cppid{audit} level are evaluated but not assumed.
  \item Checks with \cppid{default} level are evaluated but not assumed.
  \item Code for contract checks assumes that the violation handler may return.
  \item Checks with \cppid{axiom} level are assumed.
\end{itemize}

\end{enumerate}
