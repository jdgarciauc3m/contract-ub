\section{Proposed solutions}
\label{sec:proposed}

\subsection{Avoiding the undefined behavior}

This paper proposes to avoid the undefined behavior by clarifying the semantics of every
build mode in regards of both evaluation of conditions and assumption of those
conditions. For assumption of conditions the clarification needs to address the
case where continuation is disabled and when the continuation is enabled.

To define such semantics, the following simple principles are proposed to be followed:

\begin{itemize}

\item A contract that, in a given build mode, is not evaluated cannot be used
for any kind of assumption. This leads to modes where only contracts that have
been checked are used for assumptions and avoiding in this way the identified
paths towards undefined behavior.

\item Moreover, contracts that have been checked can only be assumed if the
continuation mode has been disabled. Otherwise, such assumptions cannot be made.
Note, that no special provision is needed as the application of general rules
of conditional statements would derive this behavior as illustrated in previous
sections.

\item Axiom contract are considered as if they had been evaluated when they are
	enabled. Otherwise, they are ignored.  Note that in any case, they need
		to have a valid syntax, although expressions in an axiom are
		allowed to contain invocations to declared but not defined
		functions. If an axiom contains any invocation that is declared
		but not defined the axiom is ignored.

\end{itemize}

