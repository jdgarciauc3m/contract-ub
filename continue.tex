\section{Continuation mode}
\label{sec:continue}

The ability to enable or disable the continuation mode of a translation unit as
a whole is more than it is needed. This has been agreed by several authors of
the original contracts proposal. In fact that can be safely removed as long as
there is a way to state that a specific check should continue upon violation.

\subsection{Reasons to simplify}

The original reasons~\cite{p0380r0} for including in the contract system a mode where execution
continues after running the violation handler were:

\begin{itemize}

	\item Gradual introduction of contracts in old code bases.
	\item Test of the contracts themselves.

\end{itemize}

From those, the second one is not a real issue as there are multiple solutions.
The real motivation for continuation is the gradual introduction of contracts in
old code bases. Besides that, it is also important to address the case where a library
that already has some contracts need additional contracts to be added to it.
In both cases, being able to add new contracts where a violation only triggers
some mechanism to log the violation is important.

For the very same reason, it seems that the current solution (where all contracts in
a translation unit change their behavior in case of violation from terminating the 
program to continue after violation) is not the right solution to the problem.
What it is really needed is a mechanism to log violations to new contracts, while
preexisting and well checked all contracts remain terminating upon violation.

Consequently, the first thing to do is to remove the continuation build mode
from the contract system. The build mode may be \cppid{off} (no contact is checked),
\cppid{default} (only default contracts are checked) or \cppid{audit}
(both default and audit contracts are checked). If any of those checked contracts
is violated, the violation handler is executed and when completed the program
is terminated.

To solve the need of contract violation logging a different mechanism could be
defined. However, we could also add a new contract construct to keep integration
in the contract facility. We propose to add a behavior adjective to any \cppkey{default}
or \cppkey{audit} contract to indicate that violation implies running the violation
handler and resuming execution. For that purpose we propose to use the adjective
\cppkey{continue}. In the next subsection, details are given.

\subsection{A simplified model}

Once the continuation build mode is removed a translation is controlled by the
following options:

\begin{itemize}
  \item Build mode: \cppid{off}, \cppid{default}, \cppid{audit}.
  \item Axiom mode: \cppid{off}, \cppid{on}.
\end{itemize}

In addition we propose that any precondition, postcondition or assertion may
have an optional mode. 

\begin{lstlisting}
void f(int * p) {
  g(p);
  [[assert: p!=nullptr]]; // Default. Terminates on violation.
  [[assert audit continue: *p==42]]; // Continue on violation.
  //...
}
\end{lstlisting}

Note that \cppkey{continue} does not specify a level, and in fact it is
orthogonal to it. It just changes the behavior of a contract check making it to
continue after executing the violation handler instead terminating program
execution.

The contract mode \cppkey{continue} can be applied to any \cppkey{default} or
\cppkey{audit} contract.

\begin{lstlisting}
[[assert: p!=nullptr]]; // Default contract. Terminate on violation.

[[assert continue: p!=nullptr]]; // Default contract. Continue on violation.
[[assert default continue: p!=nullptr]]; // Same as above

[[assert audit: p!=nullptr]]; // Audit contract. Terminate on violation.
[[assert audit continue: p!=nullptr]]; // Audit contract. Continue on violation
\end{lstlisting}

However, a mode is nonsens for an \cppkey{axiom} contract as they do not have
run-time semantics.

\begin{lstlisting}
[[assert axiom continue: p!=nullptr]]; // Error.
\end{lstlisting}

We selected the word \cppkey{continue} as we beleive it is the simplest way of
stating programmers intentions (continue execution after this check regardless
violation status). We find that adding adjectives (e.g. \emph{maybe}, \emph{always},
\emph{never}) are just confusing and missleading. We could also have added
a \cppkey{terminate} or \cppkey{exit} mode to state the opposite, but we beleived
that this was an unnecessary complication.
