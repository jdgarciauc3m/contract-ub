\section{Continuation mode}
\label{sec:continue}

The ability to enable or disable the continuation mode of a translation unit as
a whole is more than it is needed. This has been agreed by several authors of
the original contracts proposal. In fact that can be safely removed as long as
there is a way to state that a specific check should continue upon violation.

\subsection{Reasons to simplify}

TBD: This needs to be elaborated.

\subsection{A simplified model}

Once the continuation build mode is removed a translation is controlled by the
following options:

\begin{itemize}
  \item Build mode: \cppid{off}, \cppid{default}, \cppid{audit}.
  \item Axiom mode: \cppid{off}, \cppid{on}.
\end{itemize}

In addition we propose that any precondition, postcondition or assertion may
have an optional mode. 

\begin{lstlisting}
void f(int * p) {
  g(p);
  [[assert: p!=nullptr]]; // Default. Terminates on violation.
  [[assert audit continue: *p==42]]; // Continue on violation.
  //...
}
\end{lstlisting}

Note that \cppkey{continue} does not specify a level, and in fact it is
orthogonal to it. It just changes the behavior of a contract check making it to
continue after executing the violation handler instead terminating program
execution.

The contract mode \cppkey{continue} can be applied to any \cppkey{default} or
\cppkey{audit} contract.

\begin{lstlisting}
[[assert: p!=nullptr]]; // Default contract. Terminate on violation.

[[assert continue: p!=nullptr]]; // Default contract. Continue on violation.
[[assert default continue: p!=nullptr]]; // Same as above

[[assert audit: p!=nullptr]]; // Audit contract. Terminate on violation.
[[assert audit continue: p!=nullptr]]; // Audit contract. Continue on violation
\end{lstlisting}

However, a mode is nonsens for an \cppkey{axiom} contract as they do not have
run-time semantics.

\begin{lstlisting}
[[assert axiom continue: p!=nullptr]]; // Error.
\end{lstlisting}

We selected the word \cppkey{continue} as we beleive it is the simplest way of
stating programmers intentions (continue execution after this check regardless
violation status). We find that adding adjectives (e.g. \emph{maybe}, \emph{always},
\emph{never}) are just confusing and missleading. We could also have added
a \cppkey{terminate} or \cppkey{exit} mode to state the opposite, but we beleived
that this was an unnecessary complication.
