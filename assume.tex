\section{Assumptions and continuation mode}

\subsection{Basic example}

Consider now the following code

\begin{lstlisting}
void f(int * p) [[expects: p!= nullptr]]
{
  if (p) g();
}
\end{lstlisting}

\subsubsection{Disabled continuation and check default contracts}

If we compile with continuation mode set to \textmark{off} (the handler never returns)
and the checking level is set to \textmark{default}, the compiler can
use the information from the precondition. The generated code would be
essentially the following:

\begin{lstlisting}
void f(int * p) {
  if (p==nullptr) {
    __invoke_handler(); // Never return
  }
  else {
    g();
  }
}
\end{lstlisting}

The assumption that \cppid{p} is not \cppkey{nullptr} is derived from the
structure of the generated code and no special provision is needed to state
that the contract is assumed.

\subsubsection{Enabled continuation mode and check default contracts}

If we compile with continuation mode set to \textmark{on} (the handler might return)
and checking level set to \textmark{default}, the compiler would generate a different
code structure that woudl be essentially:

\begin{lstlisting}
void f(int * p) {
  if (p==nullptr) {
    __invoke_handler(); // May return
  }
  g();
}
\end{lstlisting}

In this case, the contract (\cppid{p!=}\cppkey{nullptr})) is checked, but if it
fails is not assumed. Again, no special provisions are needed, the behavior is the
consequence of the structure of generated code. 

\subsection{Another example}

Let's try another example:

\begin{lstlisting}
void f(int i) {
  [[assert: i==0]]
  [[assert: i>=0]]
  g();
}
\end{lstlisting}

\subsubsection{Disabled continuation and check default contracts}

In this case, the generated code would be equivalent to:

\begin{lstlisting}
void f(int i) {
  if (i!=0) __invoke_handler(); // does not return
  else {
    if (i<0) __invoke_handler(); // does not return. Never called. Optimized out
    g();
  }
}
\end{lstlisting}

The second check can only be called if the first one was successful.
But if the first was successful \cppid{i} must be \cppid{0} and the second
one will be optimized out. Note, that again this is a consequence of the 
generated code structure.

\subsubsection{Enabled continuation and check default contracts}

In this case, the generated code would be equivalent to:

\begin{lstlisting}
void f(int i)  {
  if (i!=0) __invoke_handler(); // may return
  if (i<0) __invoke_handler(); // may return. Never optimized out
  g();
}
\end{lstlisting}

Now, both checks are independent and no one can be elided.

\subsection{Consequences}

As we have shown through examples, no special provision in the standard is
needed to state our goal. Under disabled continuation mode a contract check implies
its assumption. Under enabled continuation mode a contract check does not imply
any assumption at all.
