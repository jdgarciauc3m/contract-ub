\section{Introduction}

Consider the following code:

\begin{lstlisting}
int f(int x)
  [[expects: x>0]]
{
  return 1/x;
}

int g(int x)
  [[expects audit: x==2 || x==3]]
{
  return f(x);
}
\end{lstlisting}

Under the current wording~\cite{p0542r5}, contracts that are not checked can be assumed to be
true. Consequently, we might have the following behaviours:

\begin{itemize}
  \item When build mode is \textmark{audit}, both checks are evaluated and
consequently assumed. As a consequence \cppid{f()}'s precondition will not
be evaluated as it can be proved to always be satisfied.
  \item When build mode is \textmark{default}, only the check in \cppid{f()}
would be evaluated. However, assuming the precondition in \cppid{g()} implies that
the precondition inf \cppid{f()} is always satisified. Consequently, invoking
\cppid{g()} with a value of \cppid{0} would lead to undefined behaviour.
  \item When build mode is \textmark{off}, no check evaluated. However, they are
both assumed leading again to undefined behaviour.
\end{itemize}

Additionally, when the continuation mode is \textmark{on} (continue after violation)
the assumption brings in additional undefined behavior as now we are assuming conditions
that might be false (because the failed and we returned after running the violation
handler).

This original intent of allowing assumptions was to provide an ample margin for
optimization. However, the above example is an illustration on how assumptions
may lead to undefined behavior.

The interactions of contracts and undefined behavior have been explained in
detail in~\cite{p1321r0}. However, it should be noted that the main goal of
contracts is allowing to write more correct software by helping to detect more
programming errors. Introducing new undefined behavior was an unintentional
effect that needs to be avoided.

Of course, a secondary effect of contracts is giving compilers leeway to perform
optimizations. We aim to satisfy this goal only in the cases that the first goal
is not sacrificed.
