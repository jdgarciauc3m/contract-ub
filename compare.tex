\section{Comparison with other proposals}
\label{sec:compare}

In this section, we hilight our differences with other related proposals.

\subsection{Differences with P1429R0}

In P1429R0~\cite{p1429r0} for semantics are defined (\emph{ignore}, 
\emph{assume}, \emph{check-never-continue}, and 
\emph{check-maybe-continue}). We compare here with those semantics:

\begin{itemize}
  \item The \emph{ignore} semantics is obtained in our current model
        by disabling contract checking (build mode \cppid{off}).
  \item The \emph{assume} semantics is obtained for \cppkey{axiom}
        contracts by enabling axioms. We have already expressed in this
        paper the dangers of assuming contracts for \cppkey{default}
        and \cppkey{audit}.
  \item The \emph{check-never-continue} is what we call termination and
        it is what happens for any checked \cppkey{default} or \cppkey{audit}
        contract that has no \cppkey{continue} adjective.
  \item The \emph{check-maybe-continue} is what we call \cppkey{continue} and
        it is what happens for any checked \cppkey{default} or \cppkey{audit}
        contract that has a \cppkey{continue} adjective.
\end{itemize}

P1429R0 then proposes the ability to select any of those semantics for any
contract level. Of course, all the combinations that we identified in this
paper can be expressed by those means. Below we show the equivalence:

\begin{itemize}
\item Build-mode=\cppid{off}, Axiom-mode=\cppid{off}.
\begin{itemize}
  \item \cppkey{axiom}=\emph{ignore}, \cppkey{default}=\emph{ignore}, \cppkey{audit}=\emph{ignore}.
\end{itemize}
\item Build-mode=\cppid{off}, Axiom-mode=\cppid{on}.
\begin{itemize}
  \item \cppkey{axiom}=\emph{assume}, \cppkey{default}=\emph{ignore}, \cppkey{audit}=\emph{ignore}.
\end{itemize}
\item Build-mode=\cppid{default}, Axiom-mode=\cppid{off}.
\begin{itemize}
  \item \cppkey{axiom}=\emph{ignore}, \cppkey{default}=\emph{check-never-continue}, \cppkey{audit}=\emph{ignore}.
\end{itemize}
\item Build-mode=\cppid{default}, Axiom-mode=\cppid{on}.
\begin{itemize}
  \item \cppkey{axiom}=\emph{assume}, \cppkey{default}=\emph{check-never-continue}, \cppkey{audit}=\emph{ignore}.
\end{itemize}
\item Build-mode=\cppid{audit}, Axiom-mode=\cppid{off}.
\begin{itemize}
  \item \cppkey{axiom}=\emph{ignore}, \cppkey{default}=\emph{check-never-continue}, \cppkey{audit}=\emph{check-never-continue}.
\end{itemize}
\item Build-mode=\cppid{audit}, Axiom-mode=\cppid{on}.
\begin{itemize}
  \item \cppkey{axiom}=\emph{assume}, \cppkey{default}=\emph{check-never-continue}, \cppkey{audit}=\emph{check-never-continue}.
\end{itemize}
\end{itemize}

Note that the semantics \emph{check-maybe-continue} 
(for simplicity \emph{continue}) is not in any of our combinations, 
as this semantics, in our model, is only for in-code continuing contracts.

Our most important divergence here is that we consider that individual mapping of
semantics to levels is highly dangerous and error-prone. As a mere example,
consider the following:

\begin{lstlisting}
void f(int * p)
  [[expects: p!=nullptr]]
  [[expects audit: *p > 0]];
\end{lstlisting}

If we allow any arbitrary combination we could select, for example, the following
semantics: \cppkey{default}=\emph{ignore}, \cppkey{audit}=\emph{check-never-continue}.
This combination, would lead to undefined behavior when \cppid{p}\texttt{==}\cppkey{nullptr}.

Our second divergence is that P1429R0 allows either a contract-level or a contract-mode,
where the only proposed contract-mode for now is \cppkey{check\_maybe\_continue}
that is assumed to have a \cppkey{default} level.
Instead of that, we propose de \cppkey{continue} adjective that can be applied either
to \cppkey{default} or \cppkey{audit} contracts.
We see our model as a more generalized one, and at the same time, we provide
a simpler syntax.

Last but not least, P1429R0 proposes explicit syntax for the other three semantics.
Adding a syntax for \emph{ignore} makes a contract only syntactically checked.
Adding a syntax for \emph{assume} would lead to non-ignorable assumptions.
Adding a syntax for \emph{check\_never\_continue} would introduce non-ignorable
terminating checkes. The paper proposes that most uses of those syntaxes would
be in combination with macros, which we see as a drawback.

\subsection{Differences with P1421R0}

In P1421R0~\cite{p1421r0} the five semantics from P1333R0, are cited. The main
problem with this is that despite the efforts made, it does not seem possible
to tell the difference between \emph{check-maybe-continue} and
\emph{check-always-continue}. Additonally, the rest of semantics do not
need to be defined in the standard unless they are effectively used.

Secondly, P1421R0 proposes an alternate syntax for postconditions for stating
the return value:

\begin{lstlisting}
[[ensures(r): r>0]]
\end{lstlisting}

The rationale for that is being able to tell the difference between a level and 
a variable name. We consider that not necessary as there is a fixed number of
levels and a developer may easily select any other name for the return value.

Additionally, the proposal for additional arbitrary tags does not seem justified
and is a complication that seems a redesign of contracts. This also applies to
the ability to add additional implementation defined extensions.

