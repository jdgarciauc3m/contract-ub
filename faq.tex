\section{Some questions}

\paragraph{Why do we add the axiom build mode?}
Contracts with levels \cppid{default} and \cppid{audit} are assumed only
if they have been checked. That is pure consequence of their evaluation.
However, for contracts with level \textmark{axiom}
some systems may want to take the option to still assume them when other checks
are disabled (new proposed axiom mode \textmark{on}) while in other
systems the policy may be to avoid assuming any axiom when other checks are
enabled (new proposed axiom mode \textmark{off}).

\paragraph{Would it make sense not to assume axioms when other checks are
enabled?}
An axiom is expected to be used for checks that are always true and do not need to
be checked. Usual practice should be enable assumptions on axioms.
However, in some debug modes developers might want to avoid those assumptions.
That is obtained by disabling axioms with axiom mode set to \textmark{off}.

\paragraph{Why not controlling individual semantics for each contract level?}
That would lead to some combinations that may be problematic or with surprising
behavior. Consider for example,  
enabling checking for \cppid{audit} contracts, but disabling checks for \cppid{default}
contracts. In other cases, the combination of choices might even lead again to
the undefined behavior that we are trying hard to avoid. Even worse we might
end up with the practice of needing to duplicate a predicate in more that one
assertion to guarantee it in multiple build modes.

\paragraph{Why not more build modes?}
The proposed modes seem useful for a variety of use cases. 
They also seem enough to gain
experience with the feature in C++20. After that, if needed, the catalog of
build modes might be extended in C++23.

\paragraph{Has this been implemented?}
The prototype implementation at
\url{https://github.com/arcosuc3m/clang-contracts} implements currently the new
proposed semantics for \cppkey{audit}, \cppkey{default} and \cppkey{off} as there
is no specific assumption enforcement.

\paragraph{How should axioms be taught?}
Axioms should be used only for predicates that are never wrong.
In fact, that is the mathematical notion of a logical axiom 
(\emph{a predicate that is universally true}).
Axioms are not really preconditions, postconditions, or assertions but
a portable way of spelling assumptions.
Note that an axiom should never be wrong because the consequence is
to inject undefined behavior.
