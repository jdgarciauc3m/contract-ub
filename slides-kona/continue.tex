\section{Continuation mode}

\begin{frame}[t]{Why a continuation mode}
\begin{itemize}
	\item Key \textmark{motivating} case for continuation:
    \begin{itemize}
      \item Gradual introduction of contracts in existing codebases.
      \item Adding more contracts to a library.
    \end{itemize}
  \vfill
  \item \textmark{We want}:
    \begin{itemize}
      \item Existing contracts terminate upon violation.
      \item Newly added contracts log the violation and continue.
    \end{itemize}
  \vfill
  \item \textmark{But}:
    \begin{itemize}
      \item Setting continuation for a whole translation does not give us what we want.
    \end{itemize}
\end{itemize}
\end{frame}

\begin{frame}[t]{Part I: Remove continuation mode}
\begin{itemize}
  \item \textmark{Simplification}: Remove continuation mode from translation options.

  \vfill
  \item Consequences:
    \begin{itemize}
      \item Three build modes: \cppid{off}, \cppid{default}, \cppid{audit}.
      \item Two axiom modes: \cppid{off}, \cppid{on}.
      \item All contract violations result in program termination.
    \end{itemize}
\end{itemize}
\end{frame}

\begin{frame}[t,fragile]{Part II: Add a continue syntax}
\begin{itemize}
  \item For \cppkey{default} and \cppkey{audit} contracts a \cppkey{continue} modifier
        can be specified.
    \begin{itemize}
      \item It changes the check by continuing execution after violation handler.
    \end{itemize}
\end{itemize}
	\begin{lstlisting}[escapechar=@]
[[assert: p!=nullptr]]; // Default contract. Terminates.
@\pause@
[[assert continue: p!=nullptr]]; // Default contract. Continues.
[[assert default continue: p!=nullptr]]; // Same as above
@\pause@
[[assert audit: p!=nullptr]]; // Audit contract. Terminates.
[[assert audit continue: p!=nullptr]]; // Audit contract. Continues.
\end{lstlisting}
\end{frame}
