\section{Potential for undefined behavior}

In this section we analyze some examples of possible undefined behavior.

\subsection{Example 1}

In~\cite{p1321r0} an example provided by Peter Dimov (see at
\url{https://godbolt.org/g/7TP7Mt}) with simulated contracts 
is presented.

Essentially this example translated into contracts syntax would be:

\begin{lstlisting}
void f(int x) [[expects audit: x==2]]
{
  printf("%d\n", x);
}

void g(int x) [[expects: x>=0 && x<3]]
{
  int a[3];
  a[x] = 42;
}

void foo();
void bar();
void baz();

void h(int x) [[expects: x>=1 && x<=3]]
{
  switch(x) {
    case 1: foo(); break;
    case 2: bar(); break;
    case 3: baz(); break;
  }
}

void test()
{
  int val = std::rand();

  try { f(val); /*...*/ } catch(...) { /*...*/ }
  try { g(val); /*...*/ } catch(...) { /*...*/ }
  try { h(val); /*...*/ } catch(...) { /*...*/ }
}
\end{lstlisting}

\subsubsection{Current status}

With the current definition of contracts, compiling the code with the build mode
set to \textmark{audit} is not problematic. The precondition at \cppid{f()} is
checked and it can be assumed to be true in next calls. Then, calls to \cppid{g()}
and \cppid{h()} can optimize out the contracts under the assumption that
\cppid{x} is \cppid{2}. 

However, if the build mode is set to \textmark{default}, the precondition at
\cppid{f()} would not be checked, but still assumed.  Consequently, after the
call to \cppid{f(val)} the compiler would be allowed to assume that \cppid{val}
is \cppid{2} and the preconditions of \cppid{g()} and \cppid{h()} would be
assumed to be correct and optimized out.  This would lead to undefined behavior
for calls to \cppid{g()} (access out of bounds) and a surprising outcome for
\cppid{h()} (no function is called).

\subsubsection{Avoiding unchecked assumptions}

If we change the situation to require that no assumption of unchecked contract
can be made, the outcome is quite different.

Compiling the code with the build mode set to \textmark{audit} would not be
problematic and would lead to the same outcome than with the current wording.

When the build mode is set to \textmark{default}, the precondition at
\cppid{f()} would not be checked and would not be assumed. Consequently, after
the call to \cppid{f(val)} no assumption can be made on the value of
\cppid{val}. The preconditions of \cppid{g()} and \cppid{h()} would not be
optimized out and the checks would be performed. No undefined behavior happens.

\subsection{Example 2}

This example is a variation of previous example, which is als discussed
in~\cite{p1321r0}. In this variation the precondition at function \cppid{f()} is
now moved the be an axiom.

\begin{lstlisting}
void f(int x) [[expects axiom: x==2]]
{
  printf("%d\n", x);
}
\end{lstlisting}

With the current definition of contracts, axioms are always assumed.

When the build mode is set to \textmark{default}, the precondition at
\cppid{f()} would not be checked (as it is an axiom), but still assumed.
In this case, the contract elimination is considered to be intentional as
an axiom is considered to be always true.

When the build mode is set to \textmark{off}, no precondition is checked, but
\cppid{val==2} is still assumed. 

In this specific case no change is required in the current wording.


\subsection{Example 3}

Consider now this simple example:

\begin{lstlisting}
void f(int * p) [[expects axiom: p!=nullptr]]
{
  if (p) g();
  else h();
}
\end{lstlisting}

When axioms are assumed \cppid{f()} would be optimized to always call go
\cppid{g()}. 
That is not always desirable, so we propose that axioms are not assumed when
the build mode is \textmark{off} and to add a new build mode \textmark{axiom}
where axioms are assumed (but not checked, as ever).

